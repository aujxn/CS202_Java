\documentclass[../../main.tex]{subfiles}

\begin{document}
\subsection{Introductory}
\begin{steps}
   \item Count the number of times the last item appears in the list.
      \begin{enumerate}[label=\Alph*.]
         \item Recursive prototype:\\
            \vspace{.5cm}\\
            \underline{\hspace{4cm}}count_last(\underline{\hspace{4cm}});
         \item What is the base case?\\
            \vspace{.5cm}
         \item What work should be done in the wrapper?\\
            \vspace{.5cm}
         \item What work should be done before recursive call? (if any)\\
            \vspace{.5cm}
         \item What work should be done after the recursive call? (if any)\\
            \vspace{.5cm}
      \end{enumerate}
   \item Remove all the odd numbers from the list.\\
      \begin{enumerate}[label=\Alph*.]
         \item Recursive prototype:\\
            \vspace{.5cm}\\
            \underline{\hspace{4cm}}remove_odd(\underline{\hspace{4cm}});
         \item What is the base case?\\
            \vspace{.5cm}
         \item What work should be done in the wrapper?\\
            \vspace{.5cm}
         \item What work should be done before recursive call? (if any)\\
            \vspace{.5cm}
         \item What work should be done after the recursive call? (if any)\\
            \vspace{.5cm}
      \end{enumerate}
\end{steps}

\pagebreak

\subsection{Intermediate}
\begin{steps}
   \item Copy the list. The wrapper should take another list class as an argument.
      \begin{enumerate}[label=\Alph*.]
         \item Recursive prototype:\\
            \vspace{.5cm}\\
            \underline{\hspace{4cm}}copy(\underline{\hspace{4cm}});
         \item What is the base case?\\
            \vspace{.5cm}
         \item What work should be done in the wrapper?\\
            \vspace{.5cm}
         \item What work should be done before recursive call? (if any)\\
            \vspace{.5cm}
         \item What work should be done after the recursive call? (if any)\\
            \vspace{.5cm}
      \end{enumerate}
   \item Remove all but the last two items of a linked list.
      \begin{enumerate}[label=\Alph*.]
         \item Recursive prototype:\\
            \vspace{.5cm}\\
            \underline{\hspace{4cm}}remove_except(\underline{\hspace{4cm}});
         \item What is the base case?\\
            \vspace{.5cm}
         \item What work should be done in the wrapper?\\
            \vspace{.5cm}
         \item What work should be done before recursive call? (if any)\\
            \vspace{.5cm}
         \item What work should be done after the recursive call? (if any)\\
            \vspace{.5cm}
      \end{enumerate}
   \item Remove the smallest and largest items of a linked list.
      \begin{enumerate}[label=\Alph*.]
         \item Recursive prototype:\\
            \vspace{.5cm}\\
            \underline{\hspace{4cm}}remove_big_small(\underline{\hspace{4cm}});
         \item What is the base case?\\
            \vspace{.5cm}
         \item What work should be done in the wrapper?\\
            \vspace{.5cm}
         \item What work should be done before recursive call? (if any)\\
            \vspace{.5cm}
         \item What work should be done after the recursive call? (if any)\\
            \vspace{.5cm}
      \end{enumerate}
   \item If the sum of the list is even, copy all the even data. If it is odd, copy all
      the odd data. Do this in a single traversal.
      \begin{enumerate}[label=\Alph*.]
         \item Recursive prototype:\\
            \vspace{.5cm}\\
            \underline{\hspace{4cm}}copy_sum_condition(\underline{\hspace{4cm}});
         \item What is the base case?\\
            \vspace{.5cm}
         \item What work should be done in the wrapper?\\
            \vspace{.5cm}
         \item What work should be done before recursive call? (if any)\\
            \vspace{.5cm}
         \item What work should be done after the recursive call? (if any)\\
            \vspace{.5cm}
      \end{enumerate}
\end{steps}

\subsection{Advanced}
The advanced section of this lab covers some basic information on collections, interfaces, and
initialization blocks. These are not required for your programs or exam but could potentially help out
with your Java programming assignment.
\subsubsection{Collections}
\subsubsection{Interfaces and Multiple Inheritance}
\subsubsection{Initialization Blocks}
\begin{steps}
   \item placeholder
\end{steps}

arraylist collection
multiple inheritence using interfaces
static, initialization blocks
\end{document}

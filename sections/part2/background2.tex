\documentclass[../../main.tex]{subfiles}

\begin{document}
Part two of the Java workshop is mostly practice with recursion in Java. Recursion in Java is quite
different than recursion in C++ because Java doesn't have pass by reference.
Some things to keep in mind are:
   \begin{enumerate}[label=\Alph*.]
         \item Every non-primitive type and arrays of primitives types are Java references. The references
            are most similar to C++ pointers.
         \item You can only pass these references by value. That means mutating an argument inside of
            a function only mutates it inside of the scope of that function. It remains unchanged in
            the calling routine.
         \item Remember that nice recursion trick from C++ where you passed pointers by reference? Now
            there isn't a nice way to do that.
   \end{enumerate}

There are a couple of different ways to handle this issue. One possibility is to create a class that
encapsulates the reference. This way, when you mutate the reference, it is changed within the class.
This solution is not ideal, as it requires superfluous abstractions and methods. Another option would
be to insert the reference into an array of length one. This is similar to creating a pointer to a pointer
in C++. This trick may be useful occasionally but is honestly quite kludgy and unnatural. When others
read your code they will be confused why your method takes an array when it only makes sense to pass in a
single instance of the argument type.

The recommended solution would be to return the modified value back to the calling routine. This way
everything gets linked up correctly on the backtrace through the calling stack.

Here is an example of how you could implement remove with this method for a LLL of integers:

\begin{verbatim}
   public void remove(int to_rm) throws EmptyList, NotFound
   {
      if(head == null)
         throw new EmptyList;
      this.head = remove(this.head, to_rm);
   }

   protected Node remove(Node head, int to_rm) throws NotFound
   {
      if(head == null)
         throw new NotFound(to_rm);
      if(head.get_data() == to_rm)
         return head.get_next();
      head.set_next(remove(head.get_next(), to_rm));
      return head;
   }
\end{verbatim}

abstract base class vs interface
using collections and std
static, initialization blocks

unnamed functions? closures / lambdas?

debugger basics?

\end{document}

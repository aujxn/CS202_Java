\documentclass[../../main.tex]{subfiles}

\begin{document}
\subsection{What is Java?}
Java is a platform-independent object-oriented programming language developed by Sun Microsystems.
It is commonly used for client-server web applications and is one of the most widely used
programming languages. The primary appeal of using Java is that a compiled Java program works on
any computer architecture that runs a Java Virtual Machine (JVM). This idea of portability through
limited dependencies is the primary ethos of the language but other core values include security,
simplicity, and performance.

\vspace{1em}
Java is more completely described as the Java Development Kit (JDK), a software platform made up of
various components such as the compiler, debugger, and Java Runtime Environment (JRE). The JRE is
a piece of software that contains everything needed to run a compiled Java program (called a Java
Bytecode - similar to a C++ binary) including the Java libraries, the Java Virtual Machine (JVM),
and deployment tools. The JVM is a virtual CPU that can be installed on a multitude of platforms.
This is how Java achieves platform agnosticism.

\vspace{1em}
Another benefit of the Java runtime is that it takes care of memory management through a process
called garbage collection (GC). This provides the developer the opportunity to focus on other
aspects of program design and can prevent issues like memory leaks and memory-based security
exploits. GC comes at the cost of memory consumption and inconsistent performance but for many
applications these costs are negligible.

\vspace{1em}
There are some drawbacks to large runtimes like the JRE. The most noteworthy being inconsistent
performance. It is difficult to know when the GC will run and if you are developing performance-
critical code--think airplanes or stoplights--then Java might not be the tool for you. Java uses
an incremental GC system so these inconsistencies are often minimal. Another issue is long
start up time. The runtime is so large it takes a minute to get going. This issue has been quite
the challenge for developers at Oracle to address. When you start using IntelliJ, you will
understand this issue and avoid closing the IDE at all costs. Lastly, if there is a bug in the
runtime there is very little you can do to fix your software. Although, given the incredible
complexity of modern compilers, similar issues exist within languages that don't provide runtimes.

\subsection{Why use an IDE now?}
IDE's are incredibly useful for developing large projects by providing tools like graphical
debuggers, program frameworks, continuous integration systems, and static or dynamic analysis. It
is important to start getting used to different development environments that you may encounter
throughout your education and careers. Check out the nearly 4000 plugins available for IntelliJ
from their plugin repository and you will likely find something that will contribute to your
workflow. One plugin that is particulary useful is the Vim plugin, which updates the key bindings
and provides the same multi-mode interface. It even allows you to source a vimrc for further
customization.

\subsection{Vocabulary}
Java developers use some different vocabulary that is worth being
aware of:
   \begin{itemize}
      \item Member: A Method or a Field of a class.
      \item Methods: Are functions of a class. If you use the words
         "member function" with a Java developer they might laugh at you.
      \item Fields: A piece of data in a class (like an int or a
         String). These are sometimes called variables as well but a
         variable also refers to local data where a field refers to a
         class member.
      \item Reference: Its a fancy pointer. It is very different
         from a C++ reference and much closer to a C++ pointer.
         (a C++ reference is immutable and CANNOT be null)
      \item final: A Java keyword that designates something cannot be
         changed. A final class cannot be derived from, a final
         variable is like a C++ const variable, and a final method
         cannot be overridden.
      \item static: A Java keyword that is the same as C++ but is worth
         noting. A static field is shared by all instances of the class
         and a static method is called using the class itself, not an
         instance of the class. Static methods can be invoked even when
         no objects of that class have been instantiated and can only
         operate on static fields. Static methods and variables are
         also often called class methods and class variables.
      \item super: A Java keyword to access members of a class's
         parent.
      \item Supertype (of type A): Any class or interface above A in
         the inheritance tree.
      \item this: A Java keyword to represent the instance of the class
         in which it appears. Although Java does not require it, it is
         idiomatic in Java to use this.member whenever you are
         accessing members within a class.
      \item extends: A Java keyword to declare that a class is a
         subclass of another.
      \item interface: A Java keyword used to define a collection of
         methods and constant values. Simlar to an abstract base class
         with some subtle differences. Interfaces can be used as types
         and are heavily utilized in the Java Collections.
   \end{itemize}

\subsection{Tips for Success}
Use Oracle's documentation. The Java version at the time of writing is Java
SE 13 with the most recent LTS version being Java 8.
You probably downloaded the the SE 13 version of the JDK. The API
documentation for SE 13 is found here:

\vspace{.5em}

https://docs.oracle.com/en/java/javase/13/docs/api/index.html

\vspace{.5em}

Oracle's documentation is incredibly detailed and organized. Any question you have about the language can be answered by
reading it. In the process of looking for the answer you will learn things about the language and develop skills that are not
limited to Java. The tutorials page in the documentation is also full of easy to read and detailed explanations of various topics.
Stack Overflow and other tutorial websites can be useful to gather information but be careful to assess the quality. Try using
these resources to find where in the documentation to look instead of massaging other's code examples to fit your application.

\vspace{.5em}

In extension to the previous piece of advice, consider using the Java collections. Program 5 details that you have to construct
some of your own data structures from scratch but you can use the libraries for anything else you would like. I suggest looking
at the ArrayList and HashMap classes.

\subsection{Using IntelliJ Idea}
Using an IDE like IntelliJ is helptul once you have already established strong programming skills; at
that point it can be used as a tool for organizing and working on larger projects. It provides suggestions
that can be helpful when you aren't familiar with the language or the project. However, these hints are
only valuable if you know what the IDE is suggesting and can decide if it is actually what you want to do.
IDEs can also take care of "boiler plate" code by importing templates, snippets, and project structures.

Most IDEs have community-maintained plugin repositories giving you the ability to extend its functionality.
IntelliJ has a particulary large community and ecosystem. It is important to note that Jetbrains products
are free to students with an academic email while in school. So if you would like the ultimate version it 
is available, although the community version does everything you need and is free to everyone.

\pagebreak

\subsubsection{Installing IntelliJ Idea}
\begin{steps}
   \item Navigate to www.jetbrains.com/idea/download/ and install the community version for the platform you are using. If you would like to
      install the enterprise version then fill out the student information so you can get it at no cost.
   \item Navigate to oracle.com/technetwork/java/javase/downloads and download and install the JDK for your platform. Make sure to note the
      installation directory as you may need to tell the IDE where to find this installation.
\end{steps}

\subsubsection{Creating a Project}
\begin{steps}
   \item Run IntelliJ and click create new project.
   \item If the Project SDK dropdown didn't autofill with the location of the installed JDK, click new and navigate to the previous installed
      folder.
   \item Click next. If you would like IntelliJ to generate a main class and method, check create from template and select the Command Line App
      option. Click next again.
   \item Name your project, choose where to save it, and finish the wizard.
   \item Expand the left hand side to see the directory tree. The source files are found in <Project_Name>/com.company/src/
\end{steps}

\subsubsection{Configurations and Plugins}
Like vim, many configurations can be made to make IntelliJ what you need it to be. Open file > settings and take a look what is available.
If you are used to vim key bindings, it is recomended that you install the IdeaVim plugin. Navigate to the plugins tab in settings and search
for vim. By clicking install, this new environment will be much more familiar.

\subsubsection{Adding Classes}
\begin{steps}
   \item With the source folder or any of its contents selected, classes can be added by going to file > new > Java Class.
   \item Every class in Java gets it's own source file, don't try to add multiple classes to a single file.
   \item Every class can have a Main method if you would like. This can be useful for testing features of a class.
\end{steps}

\subsubsection{Running and Debugging}
In the top right corner there are green buttons - a triangle, hammer, and bug.
\begin{steps}
   \item The hammer builds (compiles) the program
   \item The triangle runs the program - it also builds automatically if needed
   \item The bug starts the debugger
   \item If you didn't use the Command Line App template, you may need to set the entry point for your program. To do this, run > edit
      configurations will bring up a dialogue box that lets you specify which class to start from. This class must have a Main function.
      This is how you can change the entry point for your program for testing class methods.
   \item The debugger offers familiar features like step, next, and continue. To set a breakpoint, click on the line number in your source.
      A red dot should pop up and will stay there until clicked again.
\end{steps}
\end{document}

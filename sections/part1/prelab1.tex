\documentclass[../../main.tex]{subfiles}
\begin{document}
\begin{steps}
   \item Read the background material for the lab. Follow the instructions found there to install IntelliJ
      idea. Do some research and get familiar with the enviroment so you can spend your lab time learning 
      Java instead of installing and figuring out the IDE.
   \item List the data types in Java that are different than C++:\\
      \vspace{1cm}
   \item What are the differences between identifiers in Java vs. C++?\\
      \vspace{1cm}
   \item Why don't classes in Java have a semicolon afterward?\\
      \vspace{1cm}
   \item Are arrays statically or dynamically allocated in Java?\\
      \vspace{1cm}
   \item Show how to define a constant integer value equal to 10 in Java:\\
      \vspace{1cm}
   \item Show how to create an array of 10 integers in Java:\\
      \vspace{1cm}
   \item What is copied when passing an array as an argument to a function?\\
      \vspace{1cm}
   \item Show how to determine the length of an array in Java:\\
      \vspace{1cm}\\
      Does this tell you the number of items in the array or the capacity of the array?\\
      How do you determine how many items are in the array?\\
      \vspace{1cm}
   \item When passing a node reference to a function, is it possible to modify the value of that reference in the scope of the calling function?\\
      \vspace{1cm}
\end{steps}
\end{document}
